\Chapter{Bevezetés}

A fejezet célja, hogy a feladatkiírásnál kicsit részletesebben bemutassa, hogy miről fog szólni a dolgozat.
Érdemes azt részletezni benne, hogy milyen aktuális, érdekes és nehéz probléma megoldására vállalkozik a dolgozat.

Ez egy egy-két oldalas leírás.
Nem kellenek bele külön szakaszok (section-ök).
Az irodalmi háttérbe, a probléma részleteibe csak a következő fejezetben kell belemenni.
Itt az olvasó kedvét kell meghozni a dolgozat többi részéhez.


A mai rohanó világban az embereknek már nincs idejük előre berendezett lakásokat/szobákat megtekinteni, hogy aztán onnan inspirációt véve megtervezzék saját lakásukat/szobáikat, mert ez túl sok időt emésztene fel. Ehelyett különböző lakberendező szoftverek lehetnek a segítségükre. de olyan cégek is vannak amelyek professzionális lakberendező szoftvereket alkalmaznak házak szobák dizájnjának bemutatására.

Ebben a dolgozatban szeretnék bemutatni ezekből a szoftverekből egy párat elemezve a  fajtáikat, összehasonlítva a laikusok által is könnyen használható programoktól egészen a professzionális szoftverekig.

Az alapvető megoldásokat a saját programomban szeretném bemutatni. Amely egy online felületről böngészőből futtatható alkalmazás lesz. Amely a HTML5 Canvas segítségével valósítom meg.

